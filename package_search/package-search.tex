\documentclass[hyperref=unicode,ascii,xcolor=dvipsnames]{beamer}

\usepackage{fancyvrb,relsize}
\usepackage{graphicx}
\usepackage[T2A]{fontenc}
\usepackage[english]{babel}

\setbeamertemplate{navigation symbols}{}

\usebackgroundtemplate{%
 \rule{0pt}{\paperheight}%
 \hspace*{\paperwidth}%
 \makebox[0pt][r]{\includegraphics[width=20mm]{pkgsrc-long.ps}}}

%\usetheme{Boadilla}
%\usetheme{CambridgeUS}
%\usetheme{Malmoe}
%\usetheme{Singapore}
%\usetheme{boxes}

%\usecolortheme{crane}
%\usecolortheme{dove}

\usetheme[height=7mm]{Rochester}

%\usecolortheme{seagull} % very cool with \usetheme{default}
\usecolortheme[RGB={255,102,0}]{structure}
%\usefonttheme{professionalfonts}
%\useinnertheme{rectangles}

\mode<presentation>
\title{Package search in modern package managers.\\
NIH approach}
\author{Aleksey Cheusov\\
  \texttt{cheusov@NetBSD.org}}
\date{Minsk LUG meeting, 24 Nov. 2012, Belarus}

\begin{document}

%%%%%%%%%%%%%%%%%%%%%%%%%%%%%%%%%%%%%%%%%%%%%%%%%%%%%%%%%%%%%%%%%%%%%%

\newenvironment{Code}[1]%
              {\Verbatim[label=\bf{#1},frame=single,%
                  fontsize=\small,%
                  commandchars=\\\{\}]}%
              {\endVerbatim}

%\newenvironment{Code}[1]%
%               {\semiverbatim[]}%
%               {\endsemiverbatim}

\newenvironment{CodeNoLabel}%
               {\Verbatim[frame=single,%
                   fontsize=\small,%
                   commandchars=\\\{\}]}%
               {\endVerbatim}

\newenvironment{CodeNoLabelSmallest}%
               {\Verbatim[frame=single,%
                   fontsize=\footnotesize,%
                   commandchars=\\\{\}]}%
               {\endVerbatim}
\newenvironment{CodeLarge}%
               {\Verbatim[frame=single,%
                   fontsize=\large,%
                   commandchars=\\\{\}]}%
               {\endVerbatim}

%\newcommand{\prompt}[1]{\textcolor{blue}{#1}}
%\newcommand{\prompt}[1]{\textbf{#1}\textnormal{}}
\newcommand{\prompt}[1]{{\bf{#1}}}
%\newcommand{\h}[1]{\textbf{#1}}
%\newcommand{\h}[1]{\bf{#1}\textnormal{}}
\newcommand{\h}[1]{{\bf{#1}}}
\newcommand{\URL}[1]{\textbf{#1}}
\newcommand{\AutohellFile}[1]{\textcolor{red}{#1}}
\newcommand{\MKCfile}[1]{\textcolor{green}{#1}}
\newcommand{\ModuleName}[1]{\textbf{#1}\textnormal{}}
\newcommand{\ProgName}[1]{\textbf{#1}\textnormal{}}
\newcommand{\ProjectName}[1]{\textbf{#1}\textnormal{}}
\newcommand{\PackageName}[1]{\textbf{#1}\textnormal{}}
\newcommand{\MKC}[1]{\large\textsf{#1}\textnormal{}\normalsize}

%%%%%%%%%%%%%%%%%%%%%%%%%%%%%%%%%%%%%%%%%%%%%%%%%%%%%%%%%%%%%%%%%%%%%%
\begin{frame}
  \titlepage
\end{frame}

%%%%%%%%%%%%%%%%%%%%%%%%%%%%%%%%%%%%%%%%%%%%%%%%%%%%%%%%%%%%%%%%%%%%%%
\begin{frame}[fragile]
  \frametitle{Problem}
  \begin{itemize}
  \item {\bf OBS}: more than 160.000 packages for 22 distributions
  \item {\bf Debian/Linux}: about 29000 packages
  \item {\bf FreeBSD}: more than 24000 ports
  \item {\bf pkgsrc}: more than 12000 packages
  \item {\bf AltLinux}: more than 12000 packages
  \item ...
  \item but package search facilities are still very rudimental
  \end{itemize}
\end{frame}

%%%%%%%%%%%%%%%%%%%%%%%%%%%%%%%%%%%%%%%%%%%%%%%%%%%%%%%%%%%%%%%%%%%%%%

\begin{frame}[fragile]
  \frametitle{Existing practices (I)}
  \begin{itemize}
  \item {\bf apt-cache(8)} --- ``It searches the package names and the
    descriptions for an occurrence of the regular
    expression...''\copyright. Why only regexp? Why only package name
    and description? How about file names? How about {\bf apt-file(1)}?
  \item {\bf yum(8)} --- The search is limited to package names, summaries,
    descriptions, URL and regexp match.
  \item {\bf pkgng(8)/pkg-search(8) (FreeBSD)} --- Only a few search fields
    are available: one-line comment, description, origin, package name
    with/without version. The only available search strategy is regexp
  \end{itemize}
\end{frame}

%%%%%%%%%%%%%%%%%%%%%%%%%%%%%%%%%%%%%%%%%%%%%%%%%%%%%%%%%%%%%%%%%%%%%%

\begin{frame}[fragile]
  \frametitle{Existing practices (II)}
  \begin{itemize}
  \item {\bf pkgin(1) (NetBSD/pkgsrc)} --- Package name and description again plus package category. Regexp search
  \item {\bf pacman(8) (Arch Linux)} --- It is also limited to package
    name and description.
  \item pkgsrc {\bf pkg\_info(8)}, {\bf dpkg(1)}, {\bf rpm(1)}, {\bf
    pacman(1)} and others are able to answer the following questions:
    ``What installed package owns this file?'', ``What installed
    package(s) depend on this one?''
  \item {\bf aptitude(8)} --- Much better than all of the above, but it is still limited
  \end{itemize}
\end{frame}

%%%%%%%%%%%%%%%%%%%%%%%%%%%%%%%%%%%%%%%%%%%%%%%%%%%%%%%%%%%%%%%%%%%%%%

\begin{frame}[fragile]
  \frametitle{Four main questions about package search (I)}
  \begin{itemize}
  \item {\bf Where} to search
    \begin{itemize}
    \item Binary package repository
    \item Installed packages
    \item Build specifications (rpm .spec, debian/, pkgsrc tree etc.)
    \item Online resources
    \item ...
    \end{itemize}
  \item {\bf What} to match
    \begin{itemize}
    \item Package name
    \item One-line description of the package
    \item Full description of the package
    \item Package maintainer
    \item File lists
    \item Dependencies and dependants
    \item ...
    \end{itemize}
  \end{itemize}
\end{frame}

%%%%%%%%%%%%%%%%%%%%%%%%%%%%%%%%%%%%%%%%%%%%%%%%%%%%%%%%%%%%%%%%%%%%%%

\begin{frame}[fragile]
  \frametitle{Four main questions about package search (II)}
  \begin{itemize}
  \item {\bf How} to search
    \begin{itemize}
    \item Exact match
    \item Substring
    \item Regexp search
    \item Version matching (e.g. >=5.14.0<5.16)
    \item More than one query in a single request (logical AND, OR
      and NOT, for example in CNF or DNF over simple queries)
    \item ...
    \end{itemize}
  \item {\bf How much} information do we need
    \begin{itemize}
    \item Package name or identifier
    \item Short description of a package
    \item Full information about a package
    \item ...
    \end{itemize}
  \end{itemize}
\end{frame}

%%%%%%%%%%%%%%%%%%%%%%%%%%%%%%%%%%%%%%%%%%%%%%%%%%%%%%%%%%%%%%%%%%%%%%

\begin{frame}[fragile]
  \frametitle{Four main questions: conclusion}
  \begin{itemize}
  \item A list of available search fields should be {\bf variable} (package
    name, full description, maintainer etc.)
  \item A list of available search strategies should be {\bf variable}
    (exact match, regular expression etc.)
  \item High level {\bf AND}, {\bf OR} and {\bf NOT} constructions
    should be a part of {\bf frontend} while actual search should be
    done in {\bf backend}. Also, {\bf backend} determines a list of
    supported search fields and search strategies
  \end{itemize}
\end{frame}

%%%%%%%%%%%%%%%%%%%%%%%%%%%%%%%%%%%%%%%%%%%%%%%%%%%%%%%%%%%%%%%%%%%%%%

\begin{frame}[fragile]
  \frametitle{Concrete plan: backend functionality}
  \begin{itemize}
  \item Keeps a list of supported search fields and returns it on demand
  \item Keeps a list of supported search strategies and returns it on demand
  \item Given a search field, search strategies and a query, does a
    search and outputs identifiers of found packages
  \item Given a package identifier returns a description of the appropriate package
  \end{itemize}
\end{frame}

%%%%%%%%%%%%%%%%%%%%%%%%%%%%%%%%%%%%%%%%%%%%%%%%%%%%%%%%%%%%%%%%%%%%%%

\begin{frame}[fragile]
  \frametitle{Concrete plan: frontend functionality}
  \begin{itemize}
  \item Ask backend what search fields it support
  \item Ask backend what search strategies it support
  \item Call backend for actual search
  \item Provide support for AND, OR and NOT and operates on sets of package ids
  \item Provide short synonyms for search fields and strategies
  \item Do some postprocessing (formatting, coloring, GUI etc.)
  \end{itemize}
\end{frame}

%%%%%%%%%%%%%%%%%%%%%%%%%%%%%%%%%%%%%%%%%%%%%%%%%%%%%%%%%%%%%%%%%%%%%%
\begin{frame}[fragile]
  \frametitle{Package metainfo. pkg\_summary(5)}
The file pkg\_summary
contains information about each package in a binary package
repository as a list of variable-value pairs. The variables
describing different packages are separated by one empty line.

\begin{block}{Source code}
  \begin{Code}{Makefile}
PKGNAME=mrxvt-0.5.4nb6
DEPENDS=jpeg>=8nb1
DEPENDS=png>=1.5.0
COMMENT=Multi-tabbed terminal emulator with Xft support
PKGPATH=x11/mrxvt
HOMEPAGE=http://materm.sourceforge.net/
SIZE\_PKG=420641
BUILD\_DATE=2012-10-12 16:16:33 +0000
...
  \end{Code}
\end{block}
For the same purposes RFC822 is used in Debian and YAML in FreeBSD(pkgng)
\end{frame}

%%%%%%%%%%%%%%%%%%%%%%%%%%%%%%%%%%%%%%%%%%%%%%%%%%%%%%%%%%%%%%%%%%%%%%

\begin{frame}[fragile]
  \frametitle{Backends available in pkgsrc}
  \begin{itemize}
  \item {\bf pkg\_digger\_summary(1)} --- search in pkg\_summary(5) file
  \item {\bf pkg\_digger\_installed(1)} --- search in installed packages
  \item {\bf pkg\_online\_client(1)} --- search in pkg\_online database provided by dict://dict.mova.org:26280
  \end{itemize}
\end{frame}

%%%%%%%%%%%%%%%%%%%%%%%%%%%%%%%%%%%%%%%%%%%%%%%%%%%%%%%%%%%%%%%%%%%%%%

\begin{frame}[fragile]
  \frametitle{Backend pkg\_digger\_summary(1)}
  \begin{itemize}
  \item {\bf pkg\_digger\_summary} -s\\
    output available search strategies
  \item {\bf pkg\_digger\_summary} -f\\
    output available search fields
  \item {\bf pkg\_digger\_summary} field:strat:query1 [f:s:q2...]\\
    search for packages and output packages ids
  \item {\bf pkg\_digger\_summary} -1|-3|-9|-i [-r] pkgid1 [pkgid2...]\\
    output description of the package
    \begin{itemize}
    \item -1         output 1-line information about package
    \item -3         output short information about package
    \item -9|-i      output full information about package
    \item -r         raw output in pkg\_summary(5) format
    \end{itemize}
  \end{itemize}
  Other backends have the same options
\end{frame}

%%%%%%%%%%%%%%%%%%%%%%%%%%%%%%%%%%%%%%%%%%%%%%%%%%%%%%%%%%%%%%%%%%%%%%

\begin{frame}[fragile]
  \frametitle{Backend pkg\_digger\_summary(1)}
  \begin{block}{How it works}
    \begin{Code}{}
\prompt{# pkg\_digger\_summary -s}
exact   Match exactly
prefix  Match prefixes
suffix  Match suffixes
substring       Match substring
word    Match separate words
first   Match the first word
last    Match the last word
re      POSIX 1003.2 (modern) regular expressions
strfile Match the words from file
strlist Match the specified words
awk     Match using AWK expression
empty   Match an empty string
kw      "keyword" match
\prompt{#}
    \end{Code}
  \end{block}
\end{frame}

%%%%%%%%%%%%%%%%%%%%%%%%%%%%%%%%%%%%%%%%%%%%%%%%%%%%%%%%%%%%%%%%%%%%%%

\begin{frame}[fragile]
  \frametitle{Backend pkg\_digger\_summary(1)}
  \begin{block}{How it works}
    \begin{Code}{}
\prompt{# export PKG\_DIGGER\_SUMMARY=/root/summary}
\prompt{# pkg\_digger\_summary -f}
BUILD\_DATE
CATEGORIES
COMMENT
CONFLICTS
DEPENDS
DESCRIPTION
...
PKGNAME
...
PKG\_OPTIONS
PREV\_PKGPATH
PROVIDES
REQUIRES
...
\prompt{#}
    \end{Code}
  \end{block}
\end{frame}

%%%%%%%%%%%%%%%%%%%%%%%%%%%%%%%%%%%%%%%%%%%%%%%%%%%%%%%%%%%%%%%%%%%%%%

\begin{frame}[fragile]
  \frametitle{Backend pkg\_digger\_summary(1)}
  \begin{block}{How it works}
    \begin{Code}{}
\prompt{# pkg\_digger\_summary COMMENT:kw:'dns server'}
\prompt{                       COMMENT:kw:'dns proxy'}
net/maradns,maradns
net/mydns-mysql,mydns-mysql
net/mydns-pgsql,mydns-pgsql
net/nsd,nsd
net/rootprobe,rootprobe
net/unbound,unbound
net/totd,totd
\prompt{# }
    \end{Code}
  \end{block}
\end{frame}

%%%%%%%%%%%%%%%%%%%%%%%%%%%%%%%%%%%%%%%%%%%%%%%%%%%%%%%%%%%%%%%%%%%%%%

\begin{frame}[fragile]
  \frametitle{Backend pkg\_digger\_summary(1)}
  \begin{block}{How it works}
    \begin{Code}{}
\prompt{# pkg\_digger\_summary -1 net/unbound,unbound}
net/unbound        - DNS resolver and recursive server
\prompt{# pkg\_digger\_summary -3 net/unbound,unbound}
-----------------------------------------------------------
PKGNAME:        unbound-1.4.18
COMMENT:        DNS resolver and recursive server
CATEGORIES:     net
HOMEPAGE:       http://www.unbound.net/
PKGPATH:        net/unbound
DESCRIPTION:
    Unbound is an implementation of a DNS resolver.
    It provides a library...
    Homepage:
    http://www.unbound.net/

\prompt{#}
    \end{Code}
  \end{block}
\end{frame}

%%%%%%%%%%%%%%%%%%%%%%%%%%%%%%%%%%%%%%%%%%%%%%%%%%%%%%%%%%%%%%%%%%%%%%

\begin{frame}[fragile]
  \frametitle{Backend pkg\_digger\_summary(1)}
  \begin{block}{How it works}
    \begin{Code}{}
\prompt{# pkg\_digger\_summary -9r net/unbound,unbound}
PKGNAME=unbound-1.4.18
DEPENDS=ldns>=1.4
COMMENT=DNS resolver and recursive server
SIZE\_PKG=6266495
BUILD\_DATE=2012-11-09 19:13:30 +0000
CATEGORIES=net
HOMEPAGE=http://www.unbound.net/
LICENSE=modified-bsd
...
\prompt{#}
    \end{Code}
  \end{block}
\end{frame}

%%%%%%%%%%%%%%%%%%%%%%%%%%%%%%%%%%%%%%%%%%%%%%%%%%%%%%%%%%%%%%%%%%%%%%

\begin{frame}[fragile]
  \frametitle{Backend pkg\_online\_client(1)}
  \begin{block}{How it works}
    \small
    \begin{Code}{}
\prompt{# pkg\_online\_client -s}
      exact    Match headwords exactly
     prefix    Match prefixes
    nprefix    Match prefixes (skip, count)
  substring    Match substring occurring anywhere
               in a headword
     suffix    Match suffixes
         re    POSIX 1003.2 (modern) regular expressions
     regexp    Old (basic) regular expressions
    soundex    Match using SOUNDEX algorithm
        lev    Match headwords within
               Levenshtein distance one
       word    Match separate words within headwords
      first    Match the first word within headwords
       last    Match the last word within headwords
\prompt{#}
    \end{Code}
  \end{block}
\end{frame}

%%%%%%%%%%%%%%%%%%%%%%%%%%%%%%%%%%%%%%%%%%%%%%%%%%%%%%%%%%%%%%%%%%%%%%

\begin{frame}[fragile]
  \frametitle{Backend pkg\_online\_client(1)}
  \begin{block}{How it works}
    \begin{Code}{}
\prompt{# pkg\_online\_client -f}
PKGPATH
PKGNAME
PKGBASE
DEPENDS
BUILD\_DEPENDS
CONFLICTS
HOMEPAGE
COMMENT
LICENSE
ONLYFOR
NOTFOR
MAINTAINER
CATEGORIES
PLIST
DESCRIPTION
\prompt{#}
    \end{Code}
  \end{block}
\end{frame}

%%%%%%%%%%%%%%%%%%%%%%%%%%%%%%%%%%%%%%%%%%%%%%%%%%%%%%%%%%%%%%%%%%%%%%

\begin{frame}[fragile]
  \frametitle{Backend pkg\_online\_client(1)}
  \begin{block}{How it works}
    \begin{Code}{}
\prompt{# pkg\_online\_client PLIST:substring:bin/blockdiag}
graphics/py-blockdiag,py26-blockdiag
graphics/py-blockdiag,py27-blockdiag
\prompt{# pkg\_online\_client PKGBASE:prefix:dict}
textproc/dict-client,dict-client
textproc/dict-dictionaries,dict-data
textproc/dict-mueller7,dict-mueller7
textproc/dict-server,dict-server
textproc/dictem,dictem
textproc/diction,diction
wip/dict-est-rus,dict-est-rus
wip/dict-freedict-eng-ara,dict-freedict-eng-ara
wip/dict-freedict-eng-fra,dict-freedict-eng-fra
\prompt{#}
    \end{Code}
  \end{block}
\end{frame}

%%%%%%%%%%%%%%%%%%%%%%%%%%%%%%%%%%%%%%%%%%%%%%%%%%%%%%%%%%%%%%%%%%%%%%

\begin{frame}[fragile]
  \frametitle{Frontend pkg\_digger(1)}
  \begin{block}{Short synonyms for search fields}
    \begin{Code}{}
\prompt{# export PKG\_DIGGER\_SUMMARY=/root/summary}
\prompt{# pkg\_digger -f}
synonym | full name
--------------------------
     C   CATEGORIES
     c   COMMENT
     d   DESCRIPTION
     m   MAINTAINER
 empty   PKGBASE
     n   PKGNAME
     p   PKGPATH
     f   PLIST
   ...
\prompt{#}
    \end{Code}
  \end{block}
\end{frame}

%%%%%%%%%%%%%%%%%%%%%%%%%%%%%%%%%%%%%%%%%%%%%%%%%%%%%%%%%%%%%%%%%%%%%%

\begin{frame}[fragile]
  \frametitle{Frontend pkg\_digger(1)}
  \begin{block}{Short synonyms for search strategies}
    \begin{Code}{}
\prompt{# pkg\_digger -s}
 empty     exact   Match exactly
   p       prefix   Match prefixes
   u       suffix   Match suffixes
   s    substring   Match substring
   w         word   Match separate words
   f        first   Match the first word
   l         last   Match the last word
   r           re   POSIX 1003.2 (modern)
                    regular expressions
\prompt{#}
    \end{Code}
  \end{block}
\end{frame}

%%%%%%%%%%%%%%%%%%%%%%%%%%%%%%%%%%%%%%%%%%%%%%%%%%%%%%%%%%%%%%%%%%%%%%

\begin{frame}[fragile]
  \frametitle{Frontend pkg\_digger(1)}
  \begin{block}{Short synonyms for search fields/strategies}
    \begin{Code}{}
\prompt{asrock# pkg\_digger m:s:cheusov}
archivers/heirloom-tar -   Collection of standard Unix
audio/xmms-cdread      -   XMMS input plugin that read...
audio/xmms-nas         -   XMMS output plugin for Netw...
databases/libdbi       -   Database Independent Abstra...
databases/libdbi-driver-mysql - MySQL driver for libdb...
databases/libdbi-driver-pgsql - PostgreSQL driver for ...
databases/libdbi-driver-sqlite - SQLite driver for lib...
databases/libdbi-driver-sqlite3 -  SQLite3 driver for ...
devel/heirloom-getopt  -   Collection of standard Unix...
devel/heirloom-libcommon - Collection of standard Unix...
devel/heirloom-what    -   Collection of standard Unix...
devel/libjudy          -   High-performance sparse dyn...
devel/libmaa           -   General purpose data struct...
...
\prompt{#}
    \end{Code}
  \end{block}
\end{frame}

%%%%%%%%%%%%%%%%%%%%%%%%%%%%%%%%%%%%%%%%%%%%%%%%%%%%%%%%%%%%%%%%%%%%%%

\begin{frame}[fragile]
  \frametitle{Frontend pkg\_digger(1)}
{\small If neither field nor strategy was specified, fallback rules are used
(by default PKGNAME:exact, PKGPATH:exact,
PKGNAME:substring, COMMENT:word, COMMENT:substring). Frontend also
implements operations on sets (Intersection and subtraction)}
\vspace{-0.3cm}
\begin{block}{}
    \begin{Code}{}
\prompt{# pkg\_digger dns server}
net/dhisd              -   DynDNS server
net/fpdns              -   Fingerprinting DNS servers
net/maradns            -   Secure DNS server for *NIX...
net/mydns-mysql        -   MySQL-based DNS server
net/mydns-pgsql        -   PostgreSQL-based DNS serve...
net/p5-Net-DNSServer   -   Perl5 module to be used as...
net/powerdns           -   Modern, advanced and high ...
net/powerdns-recursor(pdns-recursor) -   PowerDNS res...
net/rootprobe          -   Root DNS server performanc...
net/unbound            -   DNS resolver and recursive...
...
\prompt{#}
    \end{Code}
  \end{block}
\end{frame}

%%%%%%%%%%%%%%%%%%%%%%%%%%%%%%%%%%%%%%%%%%%%%%%%%%%%%%%%%%%%%%%%%%%%%%

\begin{frame}[fragile]
  \frametitle{Finally! This approach was implemented in NIH}
NIH is a package manager for pkgsrc (pkgtools/nih)
\begin{itemize}
\item {\bf Where} to search
  \begin{itemize}
  \item {\bf nih search -b} --- search in binary repository
  \item {\bf nih search -I} --- search in installed packages
  \item {\bf nih search -o} --- search in pkg\_online database
  \item {\bf nih search -p} --- search in pkgsrc tree
  \end{itemize}
\item {\bf What} to match
  \begin{itemize}
  \item {\bf nih search -f} --- output supported search fields (-b|-I|-o|-p)
  \end{itemize}
\item {\bf How} to search
  \begin{itemize}
  \item {\bf nih search -s} --- output supported search strategies (-b|-I|-o|-p)
  \item if neither strategy nor field to search was specified we use
    fallback rules to make search easier and more efficient
  \end{itemize}
\item {\bf How much} information do we need
  \begin{itemize}
  \item {\bf -1} --- output 1-line information about found packages
  \item {\bf -3} --- output short information about found packages
  \item {\bf -9|-i} --- output full information about found packages
  \end{itemize}
\end{itemize}
\end{frame}

%%%%%%%%%%%%%%%%%%%%%%%%%%%%%%%%%%%%%%%%%%%%%%%%%%%%%%%%%%%%%%%%%%%%%%

\begin{frame}[fragile]
  \frametitle{Finally! This approach was implemented in NIH}
  \begin{itemize}
  \item Usage:
\begin{verbatim}
nih search -h
nih search [-b|-I|-o|-p] -s
nih search [-b|-I|-o|-p] -f
nih search [-b|-I|-o|-p] [-1|-3|-9|-i] query1 [query2...]
nih search [-b|-I|-o|-p] [-1|-3|-9|-i] query1 [query2...]
\end{verbatim}
\item Lines of code:
\begin{verbatim}
 212 pkg_digger_summary
  94 pkg_digger_installed
 306 pkg_digger
 330 pkg_online_client
 250 pkg_grep_summary
 254 ../share/runawk/pkg_grep_summary.awk
1446 total
\end{verbatim}
  \end{itemize}
\end{frame}

\begin{frame}[fragile]
  \frametitle{TODO}
  \begin{itemize}
  \item Reimplement {\bf pkg\_summary\_installed} for better efficiency
  \item Partially reimplement {\bf pkg\_grep\_summary} for better efficiency
  \item Currently {\bf pkg\_digger} is able to intersect and subtract sets of package ids.
    Support for for union is needed.
  \end{itemize}
\end{frame}

%%%%%%%%%%%%%%%%%%%%%%%%%%%%%%%%%%%%%%%%%%%%%%%%%%%%%%%%%%%%%%%%%%%%%%
\begin{frame}[fragile]
  \frametitle{Source code}
  \begin{itemize}
  \item pkgtools/nih\\
    (https://github.com/cheusov/pkgnih)
  \item pkgtools/pkg\_summary-utils\\
    (https://github.com/cheusov/pkg\_summary-utils)
  \item pkgtools/pkg\_online-client
  \end{itemize}
  \begin{center}
    \Huge{The End}
  \end{center}
\end{frame}

%%%%%%%%%%%%%%%%%%%%%%%%%%%%%%%%%%%%%%%%%%%%%%%%%%%%%%%%%%%%%%%%%%%%%%

\end{document}
